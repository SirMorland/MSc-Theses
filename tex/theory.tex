\section{Motorized physical sliders and their force feedback modes}

A motorized physical slider is a one-dimensional linear potentiometer, a slider, coupled with a motor. These sliders are usually manufactured to be used i.e. on mixing consoles (\cite{bourns2020}) where the motor is used just to automatically position the slider to a predefined position. However, past research has demonstrated how they can also be utilized to provide users with force feedback by making the motor try to move the slider while the user is moving it, thus applying pressure to the user's fingers (\cite{bak2015, beamish2004, berdahl-kontogeorgakopoulos2013, gillespie-rosenberg1995, papetti2018,kontogeorgakopoulos2019}). By varying the strength and direction of the motor, different force feedback modes can be created.

Student paper \textcite{kretz2004} introduces six force feedback modes for motorized sliders: \textit{Position}, \textit{Elasticity}, \textit{Friction}, \textit{Gradual}, \textit{Texture}, and \textit{Oscillation}. Aside from \textit{Friction}, these modes are rehashed in several later publications, where they are considered basic building blocks which can be combined to create new modes (\cite{jenaro2007, kretz2005, shahrokni2006}). Even though \textit{Friction} mode is missing from those publications, I have decided to include it in this study for completeness. The behaviour of each of these modes are explained in Table \ref{fjeldmodes}. \textit{Gradual} mode is also referred as \textit{Detents}, which is the term I will be using in this thesis.

\begin{table}[h!]
	\centering
	\begin{tabularx}{\textwidth}{ |c|X| }
		\hline
		\thead{Mode} & \thead{Description} \\
		\hline
		Position & Motor is turned off: the slider moves freely, and no force feedback is applied. \\
		\hline
		Elasticity & The slider emulates a rubber band: the more it is displaced from the initial position, the more it "fights back". When released the slider returns to the initial position. \\
		\hline
		Friction & The slider emulates high friction: while initiating a movement the motor applies high counter force, when released the slider stays in place. \\
		\hline
		Detents/Gradual & The slider emulates detents: while moving the slider snaps to discreet positions or "detents". \\
		\hline
		Texture & The slider emulates a rough surface: while moving the motor applies low-intensity vibration to it, thus giving an appearance of a rough surface. \\
		\hline
		Oscillation & The slider emulates damped sine movement: when released it returns to initial position following a damped sine curve. \\
		\hline
	\end{tabularx}
	\caption{Basic force feedback modes for a motorized slider (\cite{kretz2004, jenaro2007}).}
	\label{fjeldmodes}
\end{table}

\section{Electrophones and Digital Musical Instruments}

Electrophones are musical instruments that produce sound by generating an electrical signal that drives a loudspeaker (\cite{mimo2011}). They were first categorized in \textit{A Textbook of European Musical Instruments: Their Origin, History and Character} (\cite{galpin1937}) as \textit{“electrophonic instruments”} and later normalized as \textit{"electrophones"}, the fifth category of Hornbostel-Sachs classification of musical instruments (\cite{lee2019}). While electrophones include instruments such as electric guitars, where acoustic signal is amplified with electric circuitry, and pipe organs, where acoustic signal is heard when electrically controlled valve is opened, in this study I focus on electrophones that generate sound using electrical circuitry only. These instruments overlap with the definition of \glspl{dmi}.

Modern \gls{dmi} research builds on top of \citeauthor{miranda-wanderley2006}'s (\citeyear{miranda-wanderley2006}) \textit{New digital musical instruments: control and interaction beyond the keyboard}\footnote{As of writing, \textit{New digital musical instruments: control and interaction beyond the keyboard} is cited 748 times in total and 124 times in 2020 or later (Google Scholar, fetched 3.4.2023).}, which classifies \glspl{dmi} as instruments which consist of two distinct modules: a \gls{controller} and a \gls{synth}. A user interacts with the \gls{controller} (for example by pushing a button or moving a slider on it), which transforms the interaction to an electrical signal. These signals are mapped to one or more parameters of a \gls{synth}, which uses that information and its own algorithm to generate a sound wave. (\cite{miranda-wanderley2006}.)

\glspl{dmi} provide user with two types of feedback: primary feedback (visual, auditory, and tactile-kinesthetic) from the \gls{controller} and secondary feedback (auditory) from the \gls{synth}. The auditory feedback of the \gls{controller} means the mechanical noises that come from operating the \gls{controller} (such as button clicks), while the feedback from the \gls{synth} is the actual sound wave it generates, making them fundamentally different. With traditional instruments the auditory feedback is often accompanied with tactile feedback, as the user can feel the elements generating the sound (i.e., a string or a membrane) vibrating. This tactile feedback is inherently missing from \glspl{dmi} due to decoupling of \gls{controller} and \gls{synth}. Figure \ref{dmi} visualizes the flow within \gls{dmi} from user interaction to received feedback. (\cite{miranda-wanderley2006}.)

\begin{figure}[h]
	\centering
	\includegraphics[width=0.8\linewidth]{figures/dmi.png}
	\caption{Representation of a \gls{dmi} adapted from \textcite{miranda-wanderley2006}.}
	\label{dmi}
\end{figure}

\textit{Sound synthesis and sampling} (\cite{russ2009}) describes a vast number of sound synthesis techniques used in \gls{synth} part of \glspl{dmi}. \textcite{russ2009} categorizes sound synthesis techniques into five main categorizes: physical, analogue, hybrid, digital, and computer software. Analogue synthesis uses electrical circuitry to generate a continuous electrical signal for representing a sound wave, while digital synthesis uses components like microprocessors to process binary data to represent that. (\cite{russ2009}.) Note that also instruments using analogue synthesis are considered \acrlongpl{dmi}.

Particularly prevalent analogue synthesis technique is "subtractive synthesis", in which sound generation starts with harmonically rich waveform which is then modified with a filter that takes out some of the harmonies, while "analogue modeling" is a digital synthesis technique that emulates subtractive synthesis with digital circuitry (\cite{russ2009}). Modern analogue modeling synthesizers sound very close to their analogue counterparts while often offering more features and/or lower price, which have made them popular among consumers (\cite{russ2009}). In this research I focus on analogue modeling synthesis.

\section{Force feedback with \glspl{dmi}}

Force feedback in the context of \glspl{dmi} has been a subject of research at least from the late 1970s with the pioneering work of \textcite{cadoz1984} demonstrating utilizing motorized slider and piano key to give force feedback to musical gestures performed with them. Over 40 years later the topic still interests researchers as they develop new ways to incorporate force feedback into \glspl{dmi}. For example, \textcite{timmermans2020} continue the work on force feedback enabled piano keys, using a linear actuator coupled with a key to accurately replicate the force feedback action of a grand piano key, while \textcite{kirkegaard2020} introduces an open source platform for designing rotary force feedback \glspl{dmi}.

"FireFader" is another open source platform for designing force feedback \glspl{dmi} utilizing two motorized sliders (\cite{berdahl-kontogeorgakopoulos2013}). In the original research it was demonstrated to emulate a virtual mass hanging from the slider with a spring and plucking of a virtual string (\cite{berdahl-kontogeorgakopoulos2013}), while later studies have found numerous other force feedback utilizing applications for it. \textcite{papetti2018} used multiple FireFaders to form an orchestra, where each FireFader acted as a string instrument one slider being used to pluck virtual strings while other controlled the pitch or timbre of the sound. Likewise, \textcite{kontogeorgakopoulos2019} used multiple FireFaders collaboratively, connecting the sliders with virtual springs allowing performers to feel the slider movement of other performers.

\textcite{beamish2004} demonstrated "Q-Slider", a motorized slider which indicates the position of a song by automatically moving the slider accordingly and allows the user to set the position by moving the slider by hand. While moving the slider, a counter force is applied proportional to the song's amplitude on the slider's position, hence making moving the slider easier on quieter parts of a song and harder on louder parts. This was demonstrated to be a highly effective way to find different parts of a song by feeling. (\cite{beamish2004}.) \textcite{bak2015} display results of classes and workshops about sound synthesis and haptics held from 2011 to 2015. One such result was "Duojam", where two motorized sliders virtually attract each other (\cite{bak2015}).