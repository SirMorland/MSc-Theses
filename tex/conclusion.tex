\section{Results and discussion}

In this study I aimed to research what kind of effect including force feedback could have to a \gls{dmi}. My research questions were how do users find helpfulness and enjoyability of using a \gls{dmi} with force feedback compared to using one without it, and whether the inclusion of force feedback provides any measurable benefit. As can be seen from Chapter \ref{ch:results}, the results of the study were mostly inconclusive. I couldn't find any statistically significant objective measurable benefit on the context of the tasks used in the user study, but on the other hand neither hinderance. This allows me to put more weight on the subjective views, especially from the interviews. Since force feedback wasn't obstructing the usage of a \gls{dmi}, and some participants enjoyed it and found it useful, it is worthwhile to think that when well implemented it could improve the user experience for some users.

\textit{Texture} mode garnered the most positive comments from the interviews, and it also performed the best in \gls{nasa-tlx} and measured data. Participants appreciated its ability to affirm the device is working, and the applied resistance it provided. Going by these results it's somewhat safe to say a \textit{Texture} mode like feature would be a welcomed addition to a \gls{dmi}.

\textit{Detents} mode has the clear use case of reducing options, which should make both finding a value and remembering past values easier, two pros participants brought up in the interviews. However, measured data showed that in the context of the tasks it did not perform any faster or more accurate than other modes, probably the contrary of that. Also, it is likely physically more demanding to use than without force feedback, so if implemented in an actual product the strength of the force feedback should be tuned to a suitable level. The identified benefits should still make it a worthwhile consideration when designing \glspl{dmi}. If included, the user should be provided with an option to leave the slider between two detents, for example by turning off the mode.

\textit{Friction} mode was both disliked in the interviews and performed poorly in the statistical tests. Factoring in the seemingly increased \textit{physical demand}, it is apparent that too strong force feedback negatively affects the user experience. This could explain why Fjeld et al. did not include the mode in their later publications. As is the mode would not be an enjoyable or useful addition to have.

\textit{Elasticity} and \textit{Oscillation} modes were said to be fun options for performance. Due to their nature, they were not suitable for the tasks, and thus lack \gls{nasa-tlx} results and measured data to evaluate. Fun performance factors should make them ideal inclusions for a \gls{dmi}, but as participants said they should also have an option to leave the slider to a desired position without it going back to the initial one.

These force feedback modes give justification to consider motorized inputs when designing a \gls{dmi}, and having a motorized input for even one mode inevitably affords possibility to include any number of modes, as it becomes a matter of software. However, adding that motorized input comes with a cost: the circuitry of the device becomes more complex, the power requirement increases, and the added components and engineering unavoidably increases the price of the product, possibly significantly. Evaluating whether the benefits match the increase in cost is left outside of the scope of this study. If force feedback modes are added to a \gls{dmi}, the strength of those should be user configurable to account for different preferences and usage environments. Users could also be able to turn off the force feedback completely.

\section{Reliability of the study}

This study was conducted without affiliations to 3rd parties and did not receive external funding. User studies were conducted within the premises of Tampere University. During and after these studies several issues arose concerning the reliability of the overall study, as described next.

The software and hardware of HapSynth used in the tests had several issues. For many participants, most likely due to problems in hardware, the force feedback was applied only when moving the slider to one direction but not to the other. In such cases I asked the participants to imagine that the force feedback was applied similarly to both directions.

Some participants had trouble navigating HapSynth on tasks that required modifying two parameters, which shifted their attention out of feeling the force feedback. For the purposes of those tasks, it would have been beneficial to have at least one more motorized slider, so users could have just edited both parameters at the same time without needing to navigate the device.

The test setup had other issues as well. The pre-recorded sounds and the sound generated by HapSynth never matched perfectly, even if the slider was on the same exact position used when recording the samples. This could have been caused by the sampling or routing of the devices, or possibly some other parameter than what the participant was changing was slightly off from the recording. This caused some unwanted frustration with some participants. Furthermore, all edited parameters were continuous in nature, but to best see the usefulness of \textit{Detents} mode, a discrete parameter should have been used for some tasks.

Finally, the number of participants was relatively low, and did not sufficiently represent the target audience of the device. More participants would have made the statistical tests more accurate and could have revealed some significance between the force feedback modes or level out the results completely to rule out the seeming differences. More participants could also have allowed to divide them into two separate groups, novice and expert users, which could have had different results. Now only a few participants had prior experience with \glspl{dmi}, while most were students of Haptic Interaction course and thus inherently had some interest towards haptic and force feedback but not necessarily \glspl{dmi}. To get broader, more representative results, more participants should have been recruited, especially ones with more experience using \glspl{dmi}.

\section{Future work}

The objective of this research was to explore the helpfulness and enjoyability of force feedback in a \gls{dmi}. Still, many questions remain that were left outside the scope of the study, or which rose from the findings. HapSynth, the open source force feedback enabled \gls{dmi} that was developed during this thesis, provides a starting point for answering such questions.

This research utilized a motorized slider to provide force feedback, but force feedback could also be provided in numerous other ways. As mentioned in Chapter \ref{ch:methodology}, aside from slider I also explored the possibility of providing force feedback through motorized knobs or piano keys, all of which could be interesting future research topics, expanding on this thesis and works such as \textcite{kirkegaard2020} and \textcite{timmermans2020}. Different mediums could utilize the same force feedback modes presented in this work, but also provide some new ones possibly unique for those mediums. Especially the prototype of \textcite{timmermans2020} was primarily used to replicate existing piano actions, but utilizing novel force feedback modes could bring up interesting new playing methods.

Furthermore, adding a second or even more motorized sliders to the device would open avenues for new interactions and related research. Operating multiple sliders simultaneously could yield interesting results, either by using the same or different force feedback modes on each slider. Also having the sliders affect each other in some way, i.e. similarly to \textcite{kontogeorgakopoulos2019} or \textcite{bak2015}, could yield exciting results.

Finally, having two sliders perpendicularly and coupled together would provide one more degree of freedom, forming a force feedback enabled XY pad. Exploring \gls{dmi} applications for such contraption would be fascinating.