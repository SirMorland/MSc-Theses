\glspl{dmi} are instruments that produce sound by generating an electrical signal representing an audio wave, unlike traditional instruments that directly vibrate the air. Thus, they lack the tactile feedback traditional instruments provide by default, and any such feedback must be purposefully built in. Motorized inputs such as knobs and sliders are one way for \gls{dmi} designers to incorporate force feedback in their products.

The object of this study was to research what kind of effect force feedback has to the usability of a \gls{dmi}. Both users' subjective perception of usefulness and enjoyability and objective measurable benefits were explored.

The study consisted of an overview of past research of force feedback in general and in the context of musical equipment and building a prototype of a force feedback enabled \gls{dmi} based on that. Furthermore, user study with the prototype was conducted, including performing tasks  with different force feedback modes, filling \gls{nasa-tlx} questionnaires and having semi-structured interviews, and the results of those were analysed with thematic analysis and statistical tests.

The results of the study were inconclusive, interviews giving mixed opinions and few to none statistical test showed significance. However, force feedback wasn't found to hinder the performance either, and the number of participants who found at least some of the force feedback modes useful or enjoyable makes inclusion of force feedback a worthwhile consideration for \gls{dmi} designers.